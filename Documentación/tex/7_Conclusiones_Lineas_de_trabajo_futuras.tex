\capitulo{7}{Conclusiones y Líneas de trabajo futuras}


\subsection{Conclusiones}

Después de varios meses de intensa dedicación y esfuerzo en el proyecto XacoMeterII, se puede afirmar con certeza que se ha experimentado un significativo crecimiento y desarrollo tanto a nivel técnico como personal. Durante este proceso, se han adquirido habilidades y conocimientos valiosos en el campo de la programación, que han permitido mejorar las capacidades y competencias en esta área. Además, el trabajo en equipo junto con mis tutores y la resolución de desafíos han fomentado un crecimiento personal y profesional. En definitiva, el proyecto XacoMeterII ha sido una experiencia enriquecedora y formativa a nivel individual.\\

El proyecto finalmente ha cumplido con todos los objetivos marcados en un principio, añadiendo además nuevas funcionalidades que hacen a XacoMeterII una aplicación más atractiva visualmente y que permite a los usuarios interactuar con ella en cualquier dispositivo con acceso a Internet.\\

En un principio, XacoMeterII iba a ser una aplicación web que utilizase la API de Twitter para obtener los datos de una serie de bienes histórico-artísticos y mostrarlos con un gráfico al usuario.\\
Según fue avanzando el proyecto se fueron añadiendo nuevas funcionalidades, comenzando por tener un inicio de sesión para separar las funcionalidades de la aplicación y permitir solo al administrador realizar acciones en la base de datos. Se continúo con la implementación de un mapa didáctico para la búsqueda de BICs y, además de mejorar todas las ideas iniciales, se acabó ampliando el proyecto para realizar un análisis de sentimientos de cada uno de los tweets.\\

Personalmente, he de decir que he aprendido mucho sobre el manejo de las APIs, nunca había desarrollado ni había investigado sobre su uso y me ha hecho salir de mi zona de confort y poder ampliar conocimientos. He podido mejorar y practicar con conceptos ya aprendidos en la carrera, como la metodología SCRUM, que gracias a la cual he conseguido llevar un control del proyecto y una buena organización o como las bases de datos, pudiendo gestionar todos los datos y organizarlos de la mejor manera para este proyecto, y también el lenguaje Python, ya que es un lenguaje que personalmente me gusta mucho y he podido aprender más sobre él. \\
También he podido aprender a desarrollar una aplicación web desde cero, cosa que nunca había hecho, poder entender de manera práctica el modelo-vista controlador, ya que se había estudiado de manera teórica pero no se había llevado a la práctica y, aunque pueden quedar cosas que mejorar, se ha conseguido realizar todas las conexiones de manera exitosa y terminar el proyecto de manera satisfactoria.\\

Después de lo comentado, en mi opinión mi curva de mi aprendizaje ha sido buena aun con las dificultades encontradas en el proyecto.

\subsection{Líneas de trabajo futuras}
\begin{itemize}
    \item Poder crear cuentas para los usuarios y que puedan guardar sus historiales de búsquedas.
    \item Ampliar el análisis de sentimientos y evaluar la precisión de los resultados.
    \item Realizar un proceso en segundo plano para que en el caso de querer actualizar o crear la base de datos en la aplicación desplegada no termine con on \textit{'Timeout'}.
    \item Dar la opción de cambiar el idioma de la aplicación.
\end{itemize}

 
