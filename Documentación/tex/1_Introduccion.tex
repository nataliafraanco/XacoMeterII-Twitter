\capitulo{1}{Introducción}
Los gráficos estadísticos son usados en las empresas para mostrar información relevante y poder analizar qué impactos están causando.
Actualmente, una de las fuentes de información más utilizadas son las redes sociales, en las cuales, las personas pueden plasmar sus opiniones y cmpartirlas con cualquier persona del mundo. Entre las redes sociales, podemos encontrar Twitter, una de las plataformás más importantes.\\
Twitter es un servicio de microblogueo que te permite estar conectado tanto con la gente más cercana como con cualquier otra persona que esté delante de una pantalla en cualquier otro lugar del mundo. 
Este servicio ofrece a todos los usuarios realizar diferentes tipos de tareas:
\begin{itemize}
    \item \textbf{Publicar un tweet:}\\
    La aplicación permite redactar una opinión y a la vez publicarla en la red para que otros usuarios puedan verla, a esto se le llama publicar un \textit{'tweet'}.\\
    El autor decide si esa información pueden verla las personas más cercanas (sus seguidores) o por el contrario, prefiere que esa información sea pública y todos los usuarios de la red puedan leerla.
    \item \textbf{Realizar un retweet:}\\
    Existe la opción de publicar también un tweet que haya escrito otra persona sin tener que escribirlo de nuevo, haciendo referencia a esa persona, pulsando sobre el icono \textit{'Retweet'}.
    \item \textbf{Marcar un like:}\\
    Si un usuario ha visto una publicación que le ha gustado y quiere hacerselo saber al usuario sin tener que hablarle, será tan sencillo como hacer click encima del corazón y ya podrán saber todos los usuarios que vean dicho tweet que se ha realizado un \textit{'like'}.
    \item \textbf{Responder a un tweet:}\\
    También llamado \textit{'reply'}. Es la manera de contestar a una persona de manera publica para poder realizar un debate entre todos los usuarios que estén interesados en el tema del tweet.
\end{itemize}
Una vez que ya sabemos para qué se utiliza Twitter y las opciones que permite a los usuarios, podemos empezar a entender qué tiene que ver este servicio con el desarrollo de la aplicación.\\
Con este proyecto lo que se quiere conseguir es crear una aplicación web con la que podamos visualizar el impacto causado en la sociedad por el Camino de Santiago en la etapa de Castilla y León. Para esto, utilizaremos la plataforma anteriormente mencionada para recopilar información con la que más tarde podremos obtener las estadísticas reales y el interés que causan dichos bienes patrimoniales en los usuarios.
\section{Estructura de la memoria}
En la memoria podemos encontrar los siguientes apartados:
\begin{itemize}
    \item \textbf{Introducción:} se realiza una presentación del proyecto y cómo surgió la idea de realizarlo.
    \item \textbf{Objetivos del proyecto:} se exponen los objetivos tanto del proyecto como personales que se persiguen.
    \item \textbf{Conceptos teóricos:} se realiza una explicación de todos los conceptos teóricos relacionados con el desarrollo del proyecto.
    \item \textbf{Técnicas y herramientas:} se realiza una enumeración de cada una de las herramientas utilizadas y una breve explicación de su uso.
    \item \textbf{Aspectos relevantes del desarrollo:} se exponen las tareas más destacables que se han realizado durante el desarrollo del proyecto.
    \item \textbf{Conclusiones y líneas de trabajo futuras:} se realiza un listado de las posibles mejoras del proyecto y las conclusiones a las que se ha llegado tras su desarrollo.
\end{itemize}
\section{Estructura de los anexos}
También se realiza una entrega de los anexos, los cuales contienen:
\begin{itemize}
    \item \textbf{Plan de proyecto:} se realiza un análisis temporal, económico y legal de la realización del proyecto.
    \item \textbf{Especificación de requisitos:} se describen los requisitos que debe cumplir el proyecto y los objetivos esperados.
    \item \textbf{Especificación de diseño:} se explica la organización de los datos y el modelo que ha sido utilizado.
    \item \textbf{Manual del programador:} se realiza una explicación del proyecto de manera interna, tanto de los ficheros inluidos como de la instalación y el código del proyecto.
    \item \textbf{Manual del usuario:} se realiza una explicación del uso de la aplicación.
\end{itemize}
\section{Materiales del proyecto}
Además de la documentación, se entregan:
\begin{itemize}
    \item El código del proyecto.
    \item Videos explicativos del uso de la aplicación.
    \item Una máquina virtual para realizar pruebas en local.
    \item Un repositorio en GitHub
    \item El proyecto desplegado en Heroku.
\end{itemize}





