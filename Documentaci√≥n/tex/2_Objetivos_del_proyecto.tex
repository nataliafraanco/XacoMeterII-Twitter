\capitulo{2}{Objetivos del proyecto}
En este apartado, se explica de forma concisa cuáles son los objetivos de XacoMeterII tanto de carácter general como de carácter técnico.
Además, también existen unos objetivos personales que la alumna quiere cumplir así como los conocimientos que debe adquirir para cumplir con el proyecto.
\section{Objetivos generales}
\begin{enumerate}
    \item Crear de una aplicación web que pueda ser utilizada por cualquier persona conectada a la red, diferenciando a los usuarios generales de los usuarios administradores.
    \item Extraer información de la red social Twitter.
    \item Analizar los datos, incluido un análisis de sentimientos y la exportación a una base de datos.
    \item Visualizar de las estadísticas de cada BIC mediante gráficos.
\end{enumerate}
\section{Objetivos técnicos}
\begin{enumerate}
    \item Extraer de datos de la API de Twitter.
    \item Exportar datos en una base de datos ordenada por parámetros.
    \item Crear un índice de sentimientos de los contenidos de los tweets.
    \item Crear una serie de gráficos estadísticos con la herramienta Matplotlib.
    \item Crear una aplicación visual, con una interfaz clara y sencilla para cualquier usuario.
    \item La aplicación debe ser interactiva, es decir, permitir al usuario interactuar con ella ser más precisa con los resultados.
    \item Planificar el proyecto a través de la metodología SCRUM.
    \item Utilizar la herramienta Zenhub para el desarrollo del proyecto en GitHub.
\end{enumerate}
\section{Objetivos personales}
\begin{enumerate}
    \item Aprender a construir de una aplicación web Flask en Python. Es un aspecto relevante para el alumno, ya que nunca ha creado una aplicación web.
    \item Desplegar la aplicación con Heroku
    \item Aumentar los conocimientos sobre los lenguajes Python, HTML, JavaScript y SQL.
    \item Realizar conexiones con APIs.
\end{enumerate}